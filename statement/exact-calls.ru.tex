\gdef\thisproblemorigin{190421, LIII чемпионат СПбГУ}
\gdef\thisproblemauthor{Иван Казменко}
\gdef\thisproblemdeveloper{Иван Казменко}
\begin{problem}{Точное количество вызовов}
{\textsl{стандартный ввод}}{\textsl{стандартный вывод}}
%{exact-calls.in}{exact-calls.out}
{2 секунды}{512 мебибайт}{}

Рассмотрим плоское прямоугольное поле,
состоящее из $r \times c$ квадратных клеток.
Одни клетки поля свободны, а другие "--- заняты.

Робот Аркадий хочет замостить поле доминошками.
Для этого он присваивает своей внутренней переменной \texttt{counter}
значение $0$, после чего вызывает рекурсивную функцию \texttt{Tile},
представленную ниже.

\verbatiminput{program.ru.txt}

Переменная \texttt{counter} глобальная
(во всех вызовах функции одна и та же),
а переменные \texttt{row} и \texttt{col} "--- локальные
(определяются отдельно в каждом вызове).
При возврате значения сразу происходит выход из функции.
<<Вызов \texttt{Tile}>> означает рекурсивный вызов той же функции.
Клетки, на которых лежат доминошки, считаются занятыми.

Если функция \texttt{Tile} вернула Аркадию \texttt{OK},
робот успешно замостил поле доминошками.
Если же она вернула \texttt{NO}, это значит, что Аркадий
попытался замостить поле всеми возможными способами, но ничего не получилось.

Приятель Аркадия, робот Бертран, слышал от знакомых хакеров,
что Аркадий запрограммирован на какое-то интересное поведение
при условии, что замощение будет найдено ровно за $k$ вызовов
функции \texttt{Tile}, то есть значение переменной \texttt{counter}
в конце работы алгоритма в точности равно числу $k$.
Помогите этому случиться: выведите поле, которое Аркадию удастся замостить,
и при этом функция \texttt{Tile} будет вызвана ровно $k$ раз.

\InputFile

В первой строке задано целое число $k$ "---
требуемое количество вызовов функции
($1 \le k \le 10\,000\,000$).

\OutputFile

В первой строке выведите через пробел два целых числа $r$ и $c$:
размеры поля
($1 \le r, c \le 100$).
Далее, начиная со второй строки, выведите $r$ строк по $c$ символов в каждой:
прямоугольное поле.
Свободные клетки обозначаются символом <<\texttt{.}>> (точка, ASCII-код 46),
а занятые "--- символом <<\texttt{x}>> (маленькая буква икс, ASCII-код 120).
Если возможных решений несколько, выведите любое из них.

\Examples

\begin{example}
\exmp{%
\verbatiminput{001.input}%
}{%
\verbatiminput{001.output}%
}%
\exmp{%
\verbatiminput{002.input}%
}{%
\verbatiminput{002.output}%
}%
\end{example}

\end{problem}
