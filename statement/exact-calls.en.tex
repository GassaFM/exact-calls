\gdef\thisproblemorigin{190421, LIII SPbSU Championship}
\gdef\thisproblemauthor{Ivan Kazmenko}
\gdef\thisproblemdeveloper{Ivan Kazmenko}
\begin{problem}{Exact Number of Calls}
{\textsl{standard input}}{\textsl{standard output}}
%{exact-calls.in}{exact-calls.out}
{2 seconds}{512 mebibytes}{}

Consider a flat rectangular field consisting of $r \times c$ squares.
Some squares of the field are free, while others are blocked.

Arkadiy the Robot wants to cover the field with dominoes.
In order to do that, he assigns the value $0$ to his internal variable
\texttt{counter}, and then calls the recursive function \texttt{Tile}
shown below.

\verbatiminput{program.en.txt}

The variable \texttt{counter} is global
(it is the same in all calls),
while variables \texttt{row} and \texttt{col} are local
(defined separately in each call).
When a value is returned, the function terminates immediately.
A ``call to \texttt{Tile}'' means a recursive call of the same function.
The squares covered with dominoes are considered blocked.

If function \texttt{Tile} returned \texttt{OK} to Arkadiy,
the robot has successfully covered the field with dominoes.
If the function returned \texttt{NO}, it means that Arkadiy
tried all possible ways to cover the field, but was not successful.

A friend of Arkadiy, Bertrand the Robot, heard from his hacker friends
that Arkadiy is programmed to have some interesting behavior
in case the field is covered by exactly $k$ calls to \texttt{Tile},
that is, the value of the \texttt{counter} variable at the end
of the algorithm is exactly $k$.
Help to make that happen: print a field which Arkadiy will successfully cover
with dominoes in such a way that function \texttt{Tile} will be called
exactly $k$ times.

\InputFile

The first line of input contains one integer $k$:
the required number of calls
($1 \le k \le 10\,000\,000$).

\OutputFile

On the first line, print two integers $r$ and $c$ separated by a space:
the dimensions of the field
($1 \le r, c \le 100$).
Starting from the second line, print $r$ lines, each containing $c$ characters:
a rectangular field.
Free squares are denoted by ``\texttt{.}'' (dot, ASCII code 46),
and blocked squares are denoted by ``\texttt{x}''
(lowercase letter ex, ASCII code 120).
If there are several possible solutions, print any one of them.

\Examples

\begin{example}
\exmp{%
\verbatiminput{001.input}%
}{%
\verbatiminput{001.output}%
}%
\exmp{%
\verbatiminput{002.input}%
}{%
\verbatiminput{002.output}%
}%
\end{example}

\end{problem}
